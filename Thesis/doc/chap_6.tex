\Chapter{Conclusioni}


\Section{Sintesi del Percorso Sperimentale}

L’analisi sperimentale condotta ha evidenziato che il successo di una pipeline di segmentazione non dipende da un singolo fattore, bensì dall’interazione armonica tra \textbf{scelte architetturali}, \textbf{strategie di preprocessing}, \textbf{criteri di valutazione} e \textbf{robustezza del training}. Ogni esperimento ha rappresentato un passo incrementale verso una maggiore comprensione del comportamento del modello, contribuendo alla costruzione di una soluzione \textbf{affidabile}, \textbf{generalizzabile} e potenzialmente \textbf{integrabile} in contesti clinici reali.

Il percorso si è rivelato altamente \textbf{iterativo} e \textbf{evidence-based}, partendo da una configurazione di base e introducendo, con rigore sperimentale, modifiche mirate al \textbf{learning rate}, alla \textbf{funzione di loss}, al tipo di \textbf{normalizzazione}, alla \textbf{profondità architetturale}, fino a giungere a una configurazione finale profondamente ottimizzata. Particolarmente significativo è stato l’inserimento di una strategia di \textbf{validazione con trasformazioni inverse}, che ha garantito confronti fedeli tra predizioni e maschere originali, migliorando l’affidabilità delle metriche ottenute.

\Section{Risultato Finale e Implicazioni Cliniche}

Il modello finale, sviluppato \textbf{nell’esperimento 21}, ha raggiunto un \hlight{Dice score di 0.9124, arrivando fino a un Dice score di 0.9161 con il closing morfologico finale}. Dimostrando così un'elevata accuratezza predittiva. Il valore di questa performance risiede non solo nella metrica in sé, ma nella sua coerenza con una pipeline sperimentale \textbf{solida}, \textbf{riproducibile} e \textbf{scientificamente fondata}.

Dal punto di vista clinico, i risultati ottenuti hanno implicazioni rilevanti. Automatizzare la \textbf{segmentazione del tessuto mammario}, del \textbf{fibroghiandolare (FGT)} e dei \textbf{vasi sanguigni} consente di ridurre la \textbf{variabilità inter-operatore}, migliorare la \textbf{velocità di refertazione} e supportare nuove forme di \textbf{analisi quantitativa}. In particolare, la stima automatizzata della \textbf{densità mammaria} può influenzare in modo diretto la \textbf{valutazione del rischio oncologico}, abilitando percorsi diagnostici più personalizzati e precoci.

Inoltre, la presenza di maschere segmentate consente di contestualizzare meglio le eventuali lesioni, potenziando strumenti di \textbf{localizzazione assistita} e agevolando la \textbf{pianificazione preoperatoria}. Questo può essere utile, ad esempio, per identificare la \textbf{relazione tra lesioni e strutture vascolari} o valutare la \textbf{distribuzione topografica del FGT} rispetto a margini chirurgici.

\Section{Possibili Downstream Task}

Le potenzialità applicative del sistema sviluppato non si limitano alla segmentazione primaria, ma si estendono verso numerosi \textbf{downstream task}. Tra questi rientrano la possibilità di costruire \textbf{modelli predittivi di rischio} oncologico basati su \textbf{caratteristiche morfometriche} derivate dalle segmentazioni, l’integrazione del sistema all’interno di pipeline per \textbf{registrazione multimodale} (es. MRI + mammografia) e la generazione automatica di \textbf{report anatomici strutturati} per supportare la refertazione radiologica.

Un ulteriore ambito di applicazione è rappresentato dalla \textbf{quantificazione longitudinale}: in presenza di acquisizioni ripetute, il modello potrebbe consentire il monitoraggio dell’evoluzione del tessuto mammario nel tempo, supportando la valutazione dell’efficacia terapeutica o la diagnosi precoce in soggetti a rischio.

\Section{Prospettive di Sviluppo Futuro}

Tra gli sviluppi futuri più promettenti, si colloca la possibilità di estendere l’addestramento del modello a \textbf{dataset multi-istituzionali}, affrontando il problema della \textbf{generalizzazione cross-center}. Per ottenere prestazioni robuste su scanner diversi o su popolazioni eterogenee, sarà necessario introdurre meccanismi di \textbf{domain adaptation}, \textbf{normalizzazione avanzata} o \textbf{data augmentation realistica}.

Dal punto di vista architetturale, una direzione interessante è l’esplorazione di \textbf{modelli ibridi CNN-Transformer}, che combinano la capacità di catturare \textbf{dipendenze locali} con la modellazione di \textbf{relazioni globali}. In parallelo, l’ottimizzazione del modello per ambienti a bassa potenza di calcolo, mediante reti \textbf{più leggere e meno profonde}, potrebbe favorire la sua implementazione diretta in contesti clinici.


\Section{Riflessione Conclusiva}

Il presente lavoro ha rappresentato un'opportunità significativa di formazione e crescita all’interno del percorso accademico, consentendomi di confrontarmi in modo diretto con una problematica concreta e di rilevanza clinica: la \textbf{segmentazione automatica} di immagini MRI della mammella. Tale esperienza si è rivelata estremamente formativa, \hlight{non solo per il consolidamento delle competenze tecniche già acquisite in ambito accademico, ma soprattutto per lo sviluppo di un approccio sperimentale fondato su rigore metodologico, autonomia operativa e capacità critica.}

La progettazione e l’ottimizzazione della pipeline di segmentazione sono state affrontate secondo una prospettiva \textbf{graduale} e \textbf{iterativa}. A partire da una configurazione di riferimento, sono state progressivamente introdotte modifiche agli iperparametri, all’architettura del modello, alle tecniche di normalizzazione, alle funzioni di loss e alle strategie di validazione. Ogni sperimentazione ha richiesto \textbf{un’attenta analisi dei risultati}, la verifica dei dati in output, la progettazione di strumenti di debugging specifici e, spesso, la necessità di \textbf{risolvere problematiche non documentate}, tipiche del lavoro su dataset complessi e ambienti computazionali condivisi.

Tali aspetti hanno richiesto non soltanto competenze tecniche, ma anche una progressiva maturazione sul piano dell’organizzazione del lavoro sperimentale, della documentazione e della gestione degli output in modo riproducibile.

Dal punto di vista metodologico, si è trattato di un progetto orientato all’evidenza, in cui ciascuna decisione è stata motivata da osservazioni empiriche e rafforzata da test sistematici. In tal senso, la realizzazione di una pipeline di valutazione basata su \textbf{trasformazioni inverse} ha rappresentato un passaggio cruciale, migliorando sensibilmente l’affidabilità delle metriche e consentendo un confronto più aderente alla realtà clinica. La metrica finale, un \hlight{Dice score pari a 0.9161, ottenuta nel contesto dell’ultimo esperimento, è il risultato di un lungo processo di ottimizzazione e rappresenta una sintesi significativa delle scelte maturate.}

Più in generale, questa tesi ha costituito per me un’esperienza di avvicinamento concreto alla \textbf{ricerca applicata}, permettendomi di comprendere più a fondo le potenzialità e le responsabilità legate all’uso \textbf{dell’intelligenza artificiale in ambito sanitario}. È emersa in particolare l’importanza della trasparenza, della tracciabilità e della validazione nella costruzione di modelli destinati a interagire con dati clinici sensibili.

\hlight{Concludo dunque questa esperienza con la consapevolezza di aver maturato un insieme articolato di competenze, non solo sul piano tecnico, ma anche sotto il profilo metodologico e professionale.} La capacità di analizzare criticamente i risultati, documentare il processo sperimentale e adattare le soluzioni a vincoli reali rappresenta, a mio avviso, uno degli elementi più preziosi di questo percorso.
