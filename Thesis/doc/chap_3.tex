\Chapter{Dataset: Duke Breast Cancer MRI}

\Section{Composizione del Dataset}

Il dataset utilizzato proviene dal database \hlight{Duke Breast Cancer MRI} \cite{duke_breast_mri}, un archivio pubblico messo a disposizione dalla Duke University Medical Center e accessibile tramite il portale \textit{The Cancer Imaging Archive (TCIA)}. Questo dataset è stato sviluppato per promuovere la ricerca nel campo della segmentazione automatica della mammella e contiene immagini di risonanza magnetica (MRI) acquisite da pazienti con diagnosi confermata di carcinoma mammario invasivo.

Ogni studio è composto da sequenze MRI in formato \hlight{DICOM}, ottenute in posizione prona utilizzando scanner da 1.5 T o 3.0 T (prodotti da GE Healthcare o Siemens). Le immagini sono acquisite in proiezione assiale, con una risoluzione spaziale variabile compresa tra \texttt{0.6×0.6×1.0 mm\textsuperscript{3}} e \texttt{1.1×1.1×1.2 mm\textsuperscript{3}}, rendendole adatte all'elaborazione tridimensionale.

Le sequenze utilizzate nel progetto corrispondono alle immagini \textbf{T1-weighted fat-suppressed pre-contrast}, selezionate per l’elevato contrasto tra il tessuto adiposo e quello fibroghiandolare (FGT), senza l’interferenza di mezzi di contrasto. Questa scelta metodologica è coerente con quanto proposto da Lew et al. (2024), che hanno dimostrato l’efficacia di tale sequenza per la segmentazione automatica del FGT e dei vasi sanguigni.

\Subsection{Annotazioni e Ground Truth}

Una caratteristica distintiva del dataset è la disponibilità di annotazioni manuali tridimensionali, elaborate da annotatori formati e successivamente validate da radiologi esperti in senologia. Ogni caso include segmentazioni voxel-wise delle seguenti strutture:
\begin{itemize}
\item \textbf{Seno (breast)}: delimitazione del tessuto mammario escludendo parete toracica, sternale e linfonodi;
\item \textbf{Tessuto fibroghiandolare (FGT)}: zone a maggiore densità che costituiscono l’indicatore primario per la valutazione del rischio;
\item \textbf{Vasi sanguigni}: rami dell’arteria mammaria interna e dell’arteria toracica laterale, spesso iperintensi anche in assenza di contrasto.
\end{itemize}


\Subsection{Preprocessing e Integrazione con MONAI}

Durante la fase di preparazione, le immagini DICOM sono state convertite in un formato compatibile con \hlight{MONAI} (Medical Open Network for AI) \cite{cardoso2022monai}, un framework open-source ottimizzato per l'elaborazione di dati medici in ambito deep learning.

La pipeline di preprocessing ha incluso le seguenti operazioni:
\begin{itemize}
\item il \textbf{caricamento} delle immagini e delle relative maschere;
\item la \textbf{normalizzazione} dell’intensità per migliorare la stabilità del training;
\item il \textbf{ridimensionamento} delle immagini a una risoluzione uniforme per l’input nei modelli di rete neurale;
\item la conversione in \textbf{dizionari Python} contenenti l'immagine e la maschera;
\item il \textbf{salvataggio} del dataset preprocessato su disco, pronto per essere utilizzato nel training.
\end{itemize}


\Subsection{Suddivisione del Dataset}

Al fine di garantire una valutazione robusta e riproducibile dei modelli, l'intero dataset è stato suddiviso in tre sottoinsiemi:
\begin{itemize}
\item \hlight{Training set (80\%)}: utilizzato per l'addestramento del modello, comprendente la maggior parte dei dati e delle variabilità cliniche;
\item \hlight{Validation set (10\%)}: impiegato durante l’addestramento per monitorare l’overfitting e ottimizzare gli iperparametri;
\item \hlight{Test set (10\%)}: utilizzato esclusivamente per la valutazione finale delle prestazioni del modello, senza interferenze in fase di sviluppo.
\end{itemize}

