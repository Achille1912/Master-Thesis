\Chapter{Dataset: Duke Breast Cancer MRI}

\Section{Composizione del Dataset}

Il dataset utilizzato durante il tirocinio proviene dal database \hlight{Duke Breast Cancer MRI}, un archivio pubblico contenente immagini DICOM di risonanze magnetiche della mammella. Ogni caso clinico è organizzato in una struttura gerarchica che riflette l'identificativo univoco del paziente e delle relative acquisizioni.

Il dataset è composto da sequenze \hlight{DICOM} tridimensionali, ciascuna rappresentante un'acquisizione volumetrica del seno. In tutti i casi sono disponibili anche annotazioni o maschere segmentate manualmente da esperti, utilizzate come ground truth per il training dei modelli di segmentazione.

Durante la fase di preparazione, le immagini sono state convertite in formato compatibile con MONAI tramite una pipeline di preprocessing, che ha incluso:
\begin{itemize}
    \item il \textbf{caricamento} delle immagini in memoria;
    \item la standardizzazione dell'intensità e il ridimensionamento spaziale;
    \item l'organizzazione dei dati in \textbf{dizionari} contenenti sia l'immagine sia i relativi metadati;
    \item la \textbf{salvataggio} del dataset preprocessato 
\end{itemize}

Per ogni volume è stata mantenuta l'associazione tra immagine e metadati (es. spacing, orientamento, posizione del paziente), necessari per garantire una corretta elaborazione e interpretazione dei dati durante le fasi di training, validazione e inferenza.

L'intero dataset è stato infine suddiviso in tre sottoinsiemi: \hlight{training}, \hlight{validation} e \hlight{test}, rispettando la proporzione e la varietà dei casi clinici per assicurare una valutazione affidabile delle performance del modello.